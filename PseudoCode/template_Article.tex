\documentclass[]{article}
\usepackage{amsmath}
\usepackage{algorithm}
\usepackage{xcolor}
\definecolor{grey}{gray}{0.6}
\usepackage[commentColor=grey,indLines=false,rightComments=false]{algpseudocodex}
\usepackage{graphicx}

\makeatletter

\makeatother


%opening
%\title{CAVIR Pseudo Code}
%\author{Johann Schwabe}
%
%\begin{document}
%
%\maketitle
%
%
%\section{Description}
%\paragraph{Overview}
%The aim of the algorithm is, given a set of queries $\textbf{Q}$, to find for each Non-Q-Hierarchical Query $NQ_1 \in \textbf{Q}$ a set of Q-Hierarchical Query $Q_1, ..., Q_n \subset \textbf{Q}$ so that $NQ_1$ can be reduced to a Q-Hierarchical Query. A query $Q_x$ is defined by a set of relations, $Rel(Q_x) = R_{x1}, ..., R_{xn}$, it joins and by its free variables, $Var(Q_x)= var_{x1}, ..., var_{xm}$, thus also written as $Q_x(var_{x1}, ... var_{xm})= R_{x1}, ..., R_{xn}$. A relation $R_y$ is a table with schema $(var_{y1}, ... var_{yi})$, thus also written as $R_y(var_{y1}, ... var_{yi})$. A query $Q_1$ can be used to reduce a query $NQ_1$ if there exists a view $VQ_1$, which is a join of relations  $R_1, ..., R_m$ and $R_1, ..., R_m \subseteq Rel(NQ_1) \cap Rel(Q_1)$. A view is a special type of relation as it is materialized during the enumeration of its query, and all variables are free. All valid views $VQ_1^{1...v}$ of a query $Q_1$ are a subset of all permutations of $Rel(Q_1)$ of size $\geq$ 2. If a view $VQ_1$ can be used to reduce $NQ_1$, the new reduced query $NQ_1^1$ is constructed with the relations $Rel(NQ_1) \setminus Rel(VQ_1) + VQ_1$ and the free variables of $NQ_1$. After the reduction, $NQ_1^1$ could be either q-hierarchical or non-q-hierarchical. If $NQ_1^1$ is non-q-hierarchical, more reductions using different views could lead to a q-hierarchical reduction.
%
%Thus two key parts of the algorithm are: finding the valid views for q-hierarchical queries and finding compatible views and non-q-hierarichal queries.
%
%\paragraph{Valid views}
%Given the canonical variable order tree of a q-hierarchical query, its valid views can be recursively computed:
%At each node in the tree, all permutations of the childnodes $\cup$ relations at the node, of length $\geq$ 2 are valid. For example, the valid views at node $y$ in Figure \ref{VO} are any permutation with length $\geq$ 2 of $R_6, (R_5, R_3), R_1, R_2$. Thus $(R_6, R_1)$ and $(R_1, R_5, R_3)$ are valid but $(R_6, R_5)$ is not. The union of the valid views at all nodes is the set of valid views of the query.
%
%\begin{figure}[h]
%	\caption{Variable order tree for query: $Q(x,y,a,b.c,d,e) = R_6(x,y,b,d,e), R_5(x,y,a,c), R_3(x,y,a), R_1(x,y), R_2(x,y), R_0(x)$}
%	\label{VO}
%	\includegraphics[scale=0.5]{View Example}
%\end{figure}
%
%\paragraph{Finding Compatible Views}
%The brute force approach to finding compatible views for q-hierarchical queries is simple: test all views of all q-hierarchical queries with each non-q-hierarchical query. Perform the reduction wherever possible and add the resulting new queries to the corresponding sets. Repeat this process until either a reduction is found for every original query or no new reductions are found. In the worst case, there are $n!$  views per query, with $n$ being the number or relations of the query. Thus this approach is not scalable.
%A trivial improvement is remembering which queries were already compared in a previous iteration and than ignoring these combinations later on. While this significantly improves the performance, it does not solve the issue of the large number of views.
%
%A proposed improvement is to check for compatible views in decreasing size in a q-hierarchical query. Thus, in Figure \ref{VO}, first, check if a view over all relations is compatible, then if any of the views at variable $x$ are compatible and so on. If a view is found, stop early. Unfortunately, Example \ref{Ex1} is a counterexample showing that only the decomposition $Q_2^2$ can be used to reduce $Q_3$. But the view $VQ_1^{R_0,R_1,R_2}(x,y,z)$ used in $Q_2^1(...) $, is a superset of the view $VQ_1^{R_1,R_2}(x,y,z)$ used in $Q_2^2(...)$
%
%\begin{equation}
%	\begin{aligned}\label{Ex1}
%		&Q_1(...) = R_0(x,y), R_1(x,y), R_2(y,z) & Q-Hierarchical\\
%		&Q_2(...) =  R_0(x,y), R_1(x,y), R_2(y,z), R_3(x,a) & Non-Q-Hierarchical\\
%		&Q_3(...) = R_0(x,y), R_3(z,x), R_4(z, a) & Non-Q-Hierarchical\\
%		\\
%		&Q_2^1(...) = VQ_1^{R_0,R_1,R_2}(x,y,z), R_3(x,a) & Q-Hierarchical\\
%		&Q_2^2(...) = VQ_1^{R_1,R_2}(x,y,z), R_0(x,y), R_3(x,a) & Q-Hierarchical\\
%		\\
%		&Q_3^1(...) = VQ_2^2(x,y,a), R_4(z,a) & Q-Hierarchical\\
%	\end{aligned}
%\end{equation}
%
%Another proposed improvement is to stop searching for a decomposition of a query once the first decomposition is found. Unfortunately, again a counterexample exists: Example \ref{Ex2}. In a similar setting to Example \ref{Ex1}, two decompositions exist for $Q_3: Q_3^1 \& Q_3^2$, but only $Q_3^2$ can be used to reduce $Q_4$.
%\begin{equation}
%	\begin{aligned}\label{Ex2}
%		&Q_1(...) = R_1(x,y,a), R_2(y,z,a) & Q-Hierarchical\\
%		&Q_2(...) = R_2(y,z,a), R_3(z,w,a) & Q-Hierarchical\\
%		&Q_3(...) =  R_1(x,y,a), R_2(y,z,a), R_3(z,w,a), R_0(a)& Non-Q-Hierarchical\\
%		&Q_4(...) =  R_1(x,y,a), R_0(a),  R_4(y,b) & Non-Q-Hierarchical\\
%		&&\\
%		&Q_3^1(...) = VQ_1(...), R_3(z,w,a), R_0(a) & Q-Hierarchical\\
%		&Q_3^2(...) = VQ_2(...), R_1(z,w,a), R_0(a) & Q-Hierarchical\\
%		&&\\
%		&Q_4^1(...) = VQ_3^2(...), R_4(y,b) & Q-Hierarchical\\
%	\end{aligned}
%\end{equation}
%
%A third proposed improvement was to remove q-hierarchical queries from the working set once they were compared to all non-q-hierarchical queries. Another counterexample exists: Example \ref{Ex3}: After a first iteration, $Q_3^1(...)  \& Q_3^3(...) $ are found, but another iteration with $Q_1(...)$ or $Q_2(...)$ is needed to find $Q_3^3(...)$.
%
%\begin{equation}
%	\begin{aligned}\label{Ex3}
%		&Q_1(...) = R_1(x,y), R_2(y,z) & Q-Hierarchical\\
%		&Q_2(...) = R_3(a,b), R_3(b,c) & Q-Hierarchical\\
%		&Q_3(...) =  R_1(x,y), R_2(y,z), R_3(z,a), R_3(a,b) & Non-Q-Hierarchical\\
%		\\
%		&Q_3^1(...) =  VQ_1(...), R_3(z,a), R_3(a,b) & Non-Q-Hierarchical\\
%		&Q_3^0(...) =  R_1(x,y), R_2(y,z), VQ_2(...) & Non-Q-Hierarchical\\
%		\\
%		&Q_3^3(...) =   VQ_1(...), VQ_2(...) & Q-Hierarchical\\
%	\end{aligned}
%\end{equation}
%
%\paragraph{Algorithmic improvements}
%To check the compatibility of a view and a query or if a query is q-hierarchical, an \textit{isSubset} computation is needed. by expressing the sets of relations as a bitset (thus integer), high-speed binary operations can be used. For large queries with many relations and views, the integers representing the bitsets will become very large and could again reduce performance.
%
%When checking if a query is q-hierarchical, the definition requires several conditions to hold for \textbf{every pair} of variables. This can be simplified to checking every combination of join variables for the hierarchical check and every combination of join variables $\cup$ free variables for the q-check.
%
%\section{Appendix}
%\subsection{Complete}
\pagenumbering{gobble}
\newcommand\CONDITION[2]%
{\begin{tabular}[t]{@{}l@{}l@{}}
		#1&#2
	\end{tabular}%
}
\algdef{SE}[WHILE]{While}{EndWhile}[1]%
{\algorithmicwhile\ \CONDITION{#1}{\ \algorithmicdo}}%
{\algorithmicend\ \algorithmicwhile}
\algdef{SE}[FOR]{For}{EndFor}[1]%
{\algorithmicfor\ \CONDITION{#1}{\ \algorithmicdo}}%
{\algorithmicend\ \algorithmicfor}
\algdef{S}[FOR]{ForAll}[1]%
{\algorithmicforall\ \CONDITION{#1}{\ \algorithmicdo}}
\algdef{SE}[REPEAT]{Repeat}{Until}{\algorithmicrepeat}[1]%
{\algorithmicuntil\ \CONDITION{#1}{}}
\algdef{SE}[IF]{If}{EndIf}[1]%
{\algorithmicif\ \CONDITION{#1}{\ \algorithmicthen}}%
{\algorithmicend\ \algorithmicif}%
\algdef{C}[IF]{IF}{ElsIf}[1]%
{\algorithmicelse\ \algorithmicif\ \CONDITION{#1}{\ \algorithmicthen}}
\begin{document}

\begin{algorithm}
	\caption{Cavier}\label{CAVIER}
	\begin{algorithmic}[1]
		\Function{Cascade}{queries: Set[Query]}
			\State $\textit{h\_queries, non\_h\_queries} \gets \text{split(}\textit{queries}\text{)}$
			\State $\textit{new\_h\_queries} \gets \text{set()}$
			\State $\textit{new\_non\_h\_queries} \gets \text{set()}$
			\State $\textit{past\_comparisons} \gets \text{set()}$
			\While{$\textbf{True}$}
				\For{$\textit{h\_query} \text{ in } \textit{h\_queries}$}
					\Comment{for every pair of h- and non-h-queries}
					\For{$\textit{non\_h\_query} \text{ in } \textit{non\_h\_queries}$}
						\If{$\text{(}\textit{h\_query, non\_h\_query}\text{) in } \textit{past\_comparisons}$}
							\State \textbf{continue}
							\Comment{skip already searched comparisons}
						\Else
							\State{$\textit{past\_comparisons}\text{.add((}\textit{h\_query, non\_h\_query}\text{))}$}
						\EndIf
						\If{$\textit{non\_h\_query}\text{.origin in }\textit{h\_query}\text{.dependant\_on()}$}
							\State \textbf{continue}
							\Comment{Skip if h-query depends on non-h-query}
						\EndIf
						\State
						\Comment{Check if h-query and non-h-query can be cascaded: Whether a sub-query of the non-h-query can be replaced by the h-query}
						
						\Comment{is\_homomorphism checks whether the joinconditions in the h\_query are a subset of those in the non\_h\_query even if the variable names don't match}
						
						\Comment{free\_variable\_present verifies whether the variable in the h\_query that are free variables of the non\_h\_query and all join variable between the relation of the h\_query and the other relations of the non\_h\_query are free variables of the h\_query}
						\If{$\text{relations\_match(}\textit{h\_query, non\_h\_query}\text{)} $\\$ \textbf{and}  \text{ is\_homomorphism(}\textit{h\_query, non\_h\_query}\text{)}$ \\ $\textbf{and}   \text{ free\_variables\_present(}\textit{h\_query, non\_h\_query}\text{)}$}

							\State \textit{new\_query} \text{ = replace\_views(}\textit{h\_query, non\_h\_query}\text{)} 
							\State \Comment{Add new query to the search queue}
							\If{$\textit{new\_query}\text{.is\_q\_hierarchical()}$}
									\State $\textit{new\_h\_queries}\text{.add(}\textit{new\_query}\text{)}$
							\Else
									\State $\textit{new\_non\_h\_queries}\text{.add(}\textit{new\_query}\text{)}$
							\EndIf

						\EndIf
		
					\EndFor
				\EndFor
				\State 
				\Comment{complete\_solution selects a set of queries from h\_queries so that for each query in queries a single q-hierarchical reduction exists and each query is only dependant on queries within the selected set}
				\If{($\textit{solution} \gets \text{complete\_solution(}\textit{h\_queries} \cup \textit{new\_h\_queries} ) )\neq None$}
					\State \textbf{return} \textit{solution}
				\EndIf
				\If{$\text{empty(}\textit{new\_h\_queries}) \wedge \text{empty(}\textit{new\_non\_h\_queries}) $}
					\State	\textbf{return None}
				\EndIf
				\State
				\Comment{Solution not found, add new queries to the search queue}
				\State $\textit{h\_queries}\text{.extend(}\textit{new\_h\_queries})$
				\State $\textit{non\_h\_queries}\text{.extend(}\textit{new\_non\_h\_queries}\text{)}$
			\EndWhile
		\EndFunction
	\end{algorithmic}
\end{algorithm}

\begin{section}{Definition of Cascading Q Hierarchical}
	A set of conjunctive queries is considered cascading q-hierarchical if there exists a partial order of the queries, such that they can be evaluated with constant update time under single-tuple updates, and the query results can be enumerated according to the partial order with constant time delay.
	
\end{section}

\begin{section}{Proof of Completeness}
	Contradiction: There exists a set of q-hierarchical queries Q and a set of non-q-hierarchical queries NQ that together are Cascading Q Hierarchical but Cavier cannot find the partial order. There must exist a single non-q-hierarchical query $nq$ that is dependant on a q-hierarchical query $q$. As circular ordering is invalid at least on such pair must exist.
	Cavier will have compared it a first time, thus it will not skip it in line 9.
	For $q$ to be valid in replacing parts of $nq$, $q$ must not contain a view from $nq$ (non-circular), thus line 12 must be false.
	As the set of $nq$ and $q$ is defined to be cascading q-hierarchical, the relations from $q$ must be contained in $nq$, it must be a homomorphism and the necessairy free variables for $nq$ must be present in $q$. Thus Cavier will find the partial order between $nq$ and $q$ contradicting the predicate.
	Together with the prove of termination, this proves completness.
	
\end{section}
\begin{section}{Proof of Termination}
	Each query can only depend on each other query at most once. Thus there is only a finite number of intermediate queries and thus eventually no new queries are constructed and the algorithm will terminate	
\end{section}
\begin{section}{Proof of Correctness}
	Each replacement of relations in a query with the enumeration of another query produces a valid query, if the constraints of homomorphism and the free variables hold. As repeated replacements are valid, the resulting queries have to also be valid.
\end{section}

\begin{section}{Proof of Constant enumeration time of a set of Cascading Q-hierarichical Queries}
A set of cascading Q-hierarchical queries can be enumerated in O(1) by executing them as follows:
Following the partial order, a query can either be executed individually (not dependent on other queries) or by reusing a view on a previously executed query. By the Proof of Correctness, each query is Q-hierarchical when replacing relations with views on enumerated queries. To ensure that the enumerated query does not contain tuples that cannot be joined with the rest of the  remaining relations, the view is filtered during enumeration by joining it with the indicator projection of the next join with one of the remaining unreplaced relations. As this join can be executed in O(1), the asymptotic complexity remains the same. 
If a query is based on multiple Views of enumerated queries, only for the second enumerated view filtering using indicator projections is possible. As the unfiltered enumerated query allows constant time lookup, O(1) can be maintained.
\end{section}

\begin{section}{Proof of  constant update time of a set of Cascading Q-hierarichical Queries}
	On an update to a base relation, not all queries are updated eagerly. The updates are propagated until an enumeration step would be needed.
\end{section}


\end{document}

